\documentclass{amsart}

\usepackage[T1]{fontenc}
\usepackage{enumerate, amsmath, amsfonts, amssymb, amsthm, mathrsfs, wasysym, graphics, graphicx, xcolor, url, hyperref, hypcap, xargs, multicol, pdflscape, multirow, hvfloat, array, ae, aecompl, pifont, mathtools, a4wide, float, blkarray, overpic, nicefrac}
\usepackage[bb=boondox]{mathalfa}%allows for mathbb 0 and 1
\usepackage[shortlabels, inline]{enumitem}
\usepackage[noabbrev,capitalise]{cleveref}
\usepackage[normalem]{ulem}
\usepackage{marginnote}
\hypersetup{colorlinks=true, citecolor=darkblue, linkcolor=darkblue}
\usepackage[all]{xy}
\usepackage{tikz}
\usepackage{tikz-cd}
%\usepackage{tkz-graph}
\usetikzlibrary{trees, decorations, decorations.pathmorphing, decorations.markings, decorations.shapes, shapes, arrows, matrix, calc, fit, intersections, patterns, angles}
\graphicspath{{figures/}{figures/diagonals/}{figures/walks/}{figures/tubes/}{figures/blocks/}}
\makeatletter\def\input@path{{figures/}}\makeatother
\usepackage{caption}
\captionsetup{width=\textwidth}
\usepackage[export]{adjustbox}

%%%%%%%%%%%%%%%%%%%%%%%%%%%%%%%%%%%%%%

% theorems
\newtheorem{theorem}{Theorem}[section]
\newtheorem{corollary}[theorem]{Corollary}
\newtheorem{proposition}[theorem]{Proposition}
\newtheorem{lemma}[theorem]{Lemma}
\newtheorem{conjecture}[theorem]{Conjecture}
\newtheorem*{theorem*}{Theorem}%[section]

\theoremstyle{definition}
\newtheorem{definition}[theorem]{Definition}
\newtheorem{example}[theorem]{Example}
\newtheorem{remark}[theorem]{Remark}
\newtheorem{question}[theorem]{Question}
\newtheorem{notation}[theorem]{Notation}
\newtheorem{assumption}[theorem]{Assumption}
\newtheorem{convention}[theorem]{Convention}

\crefname{equation}{Equation}{Equations}

% math special letters
\newcommand{\R}{\mathbb{R}} % reals
\newcommand{\Q}{\mathbb{Q}} % rationals
\newcommand{\N}{\mathbb{N}} % naturals
\newcommand{\Z}{\mathbb{Z}} % integers
\newcommand{\C}{\mathbb{C}} % complex
\newcommand{\I}{\mathbb{I}} % set of integers
\newcommand{\HH}{\mathbb{H}} % hyperplane
\newcommand{\K}{k} % field
\newcommand{\f}[1]{{\mathfrak{#1}}} % mathfrak letters
\renewcommand{\c}[1]{{\mathcal{#1}}} % call letters
\renewcommand{\b}[1]{{\boldsymbol{#1}}} % bold letters
\newcommand{\h}{\widehat} % hat letters

% math commands
\newcommand{\set}[2]{\left\{ #1 \;\middle|\; #2 \right\}} % set notation
\newcommand{\smallset}[2]{\{ #1 \;\big|\; #2 \}} % small set notation
\newcommand{\bigset}[2]{\big\{ #1 \;\big|\; #2 \big\}} % big set notation
\newcommand{\Bigset}[2]{\Big\{ #1 \;\Big|\; #2 \Big\}} % Big set notation
\newcommand{\setangle}[2]{\left\langle #1 \;\middle|\; #2 \right\rangle} % set notation
\newcommand{\ssm}{\smallsetminus} % small set minus
\newcommand{\dotprod}[2]{\langle \, #1 \; | \; #2 \, \rangle} % dot product
\newcommand{\bigdotprod}[2]{\big\langle \, #1 \; \big| \; #2 \, \big\rangle} % dot product
\newcommand{\symdif}{\,\triangle\,} % symmetric difference
% \newcommand{\one}{{1\!\!1}} % the all one vector
\newcommand{\one}{{\mathbb{1}}} % the all one vector
\newcommand{\zero}{{\mathbb{0}}} % the all zero vector

\newcommand{\eqdef}{\mbox{\,\raisebox{0.2ex}{\scriptsize\ensuremath{\mathrm:}}\ensuremath{=}\,}} % :=
\newcommand{\defeq}{\mbox{~\ensuremath{=}\raisebox{0.2ex}{\scriptsize\ensuremath{\mathrm:}} }} % =:
\newcommand{\simplex}{\triangle} % simplex
\renewcommand{\implies}{\Rightarrow} % imply sign
\newcommand{\transpose}[1]{{#1}^T} % transpose matrix

% operators
\DeclareMathOperator{\conv}{conv} % convex hull
\DeclareMathOperator{\vect}{vect} % linear span
\DeclareMathOperator{\cone}{cone} % cone hull
\DeclareMathOperator{\sign}{sign} % sign
\DeclareMathOperator{\suppmin}{\mu} % support minimal

% others
\newcommand{\ie}{\textit{i.e.}~} % id est
\newcommand{\eg}{\textit{e.g.}~} % exempli gratia
\newcommand{\Eg}{\textit{E.g.}~} % exempli gratia
\newcommand{\apriori}{\textit{a priori}} % a priori
\newcommand{\viceversa}{\textit{vice versa}} % vice versa
\newcommand{\versus}{\textit{vs.}~} % versus
\newcommand{\aka}{\textit{a.k.a.}~} % also known as
\newcommand{\perse}{\textit{per se}} % per se
\newcommand{\ordinal}{\textsuperscript{th}} % th for ordinals
\newcommand{\ordinalst}{\textsuperscript{st}} % st for ordinals
\definecolor{darkblue}{rgb}{0,0,0.7} % darkblue color
\definecolor{green}{RGB}{57,181,74} % green color
\definecolor{violet}{RGB}{147,39,143} % violet color
\newcommand{\red}{\color{red}} % red command
\newcommand{\blue}{\color{blue}} % blue command
\newcommand{\orange}{\color{orange}} % orange command
\newcommand{\green}{\color{green}} % green command
\newcommand{\darkblue}{\color{darkblue}} % darkblue command
\newcommand{\defn}[1]{\textsl{\darkblue #1}} % emphasis of a definition
\newcommand{\para}[1]{\medskip\noindent\textsc{#1.}} % paragraph
\renewcommand{\topfraction}{1} % possibility to have one page of pictures
\renewcommand{\bottomfraction}{1} % possibility to have one page of pictures
\newcommand{\ex}{_{\textrm{exm}}} % examples
\newcommand{\pa}{_{\textrm{pa}}} % path
\newcommand*\circled[1]{\tikz[baseline=(char.base)]{\node[shape=circle, draw, inner sep=1.5pt, scale=.7] (char) {#1};}}
\newcommand{\compactVectorD}[2]{\begin{bmatrix} #1 \\ #2 \end{bmatrix}}
\newcommand{\compactVectorT}[3]{\begin{bmatrix} #1 \\[-.1cm] #2 \\[-.1cm] #3 \end{bmatrix}}

% marginal comments
\usepackage{todonotes}
\newcommand{\emily}[1]{\todo[color=red!30]{#1 \\ \hfill --- E.}}
\newcommand{\jc}[1]{\todo[color=green!30]{#1 \\ \hfill --- J.-C.}}
\newcommand{\vincent}[1]{\todo[color=blue!30]{#1 \\ \hfill --- V.}}

% geometry
\newcommandx{\polytope}[1][1 = P]{\mathsf{#1}} % polytope
\newcommandx{\Asso}[1][1=n]{\polytope{Asso}(#1)} % associahedron
\newcommandx{\Zono}[1][1=\digraph]{\polytope{Zono}(#1)} % zonotope
\newcommandx{\Quotientope}[1][1=\equiv]{\polytope{Quot}(#1)} % quotientope
\newcommandx{\QuotientFan}[1][1=\equiv]{\c{F}(#1)} % quotient fan


%%%%%%%%%%%%%%%%%%%%%%%%%%%%%%%%%%%%%%

% formating the part command
\makeatletter
\def\part{\@startsection{part}{1}%
\z@{.7\linespacing\@plus\linespacing}{.8\linespacing}%
{\LARGE\sffamily\centering}}
%\@addtoreset{section}{part}
\makeatother
\renewcommand{\thepart}{\Roman{part}}
%\renewcommand{\thesection}{\arabic{part}.\arabic{section}}

% formating the table of contents
\setcounter{tocdepth}{4}
\makeatletter
\def\l@section{\@tocline{1}{5pt}{0pc}{}{}}
\makeatother
\let\oldtocpart=\tocpart
\renewcommand{\tocpart}[2]{\sc\large\oldtocpart{#1}{#2}}
\let\oldtocsection=\tocsection
\renewcommand{\tocsection}[2]{\bf\oldtocsection{#1}{#2}}
\let\oldtocsubsubsection=\tocsubsubsection
\renewcommand{\tocsubsubsection}[2]{\quad\oldtocsubsubsection{#1}{#2}}

%%%%%%%%%%%%%%%%%%%%%%%%%%%%%%%%%%%%%%

\title[Separating trees and simple comgruences]{Separating trees and simple congruences of the weak order}
%\title{Poset associahedra as sections of graph associahedra}

\thanks{
VP was partially supported by the French project CHARMS (ANR~19\,CE40\,0017), by the French\,--\,Austrian project PAGCAP (ANR~21\,CE48\,0020 \& FWF I 5788), and by the Spanish projects PID2019-106188GB-I00 and PID2022-137283NB-C21 of MCIN/AEI/10.13039/501100011033.
}

\author{Emily Barnard}
\address[Emily Barnard]{DePaul University, Chicago, U.S.A.}
\email{e.barnard@depaul.edu}
\urladdr{\url{https://emilybarnard.github.io}}

\author{Jean-Christophe Novelli}
\address[Jean-Christophe Novelli]{Université Gustave Eiffel, Marne-la-Vallée, France}
\email{novelli@univ-mlv.fr}
\urladdr{\url{https://igm.univ-mlv.fr/~novelli/}}

\author{Vincent Pilaud}
\address[Vincent Pilaud]{Universitat de Barcelona, Spain}
\email{vincent.pilaud@ub.edu}
\urladdr{\url{https://www.ub.edu/comb/vincentpilaud/}}

%%%%%%%%%%%%%%%%%%%%%%%%%%%%%%%%%%%%%%

\begin{document}

\begin{abstract}
\end{abstract}

\maketitle

%\tableofcontents

%%%%%%%%%%%%%%%%%%%%%%%%%%%%%%%%%%%%%%%

\section{Introduction}
\label{sec:introduction}

%%%%%%%%%%%%%%%%%%%%%%%%%%%%%%%%%%%%%%%

\section{Separating trees}
\label{sec:separatingTrees}

\vincent{parent, children, ancestor, descendant}

\begin{definition}
\label{def:separatingTree}
A \defn{separating tree} is an oriented tree on~$[n]$ such that
\begin{itemize}
\item each node has at most two parents and at most two children,
\item node~$j$ with two parents (resp.~children) separates its two ancestor (resp.~descendant) subtrees, meaning that all nodes in the left subtree are smaller than~$j$ while all nodes in the right subtree are larger than~$j$.
\end{itemize}
\end{definition}

To represent separating trees, ...
\vincent{todo}
\vincent{add a figure with examples of separating trees.}

\begin{remark}
Readers familiar with~\cite{PilaudPons-permutrees} might wonder what differs between permutrees and separating trees.
The subtle distinction is that empty ancestor or descendant subtrees are allowed in permutrees and not in separating trees.
For instance, a in a separating tree, it is impossible to say that a node has two children, one of which is empty.
\end{remark}

We gather in this section some structural lemmas about separating trees.

\begin{lemma}
\label{lem:separatingTree1}
Consider a separating tree~$T$ and let~$1 \le i < j < k \le n$. Then
\begin{itemize}
\item if~$i \leftarrow k$ is an arc of~$T$ , then there is a directed path in~$T$ joining either~$i$ to~$j$, or~$j$ to~$k$,
\item if~$i \to k$ is an arc of~$T$ , then there is a directed path in~$T$ joining either~$j$ to~$i$, or~$k$ to~$j$.
\end{itemize}
%A symmetric statement holds for an arc~$i \leftarrow k$.
\end{lemma}

\begin{proof}
\vincent{todo}
\end{proof}

\begin{corollary}
\label{coro:separatingTree1}
In a separating tree, there is no edge passing above a node with two parent or below a node with two children.
\end{corollary}

In other words, we can draw a separating red wall above each node with two parents and below each node with two children, and the edges of the separating tree will never cross the red walls.

\begin{lemma}
\label{lem:separatingTree2}
Consider a node~$j$ with two parents (resp.~children) in a separating tree~$T$, and denote by~$M$ the maximum of the left ancestor (resp.~descendant) subtree of~$j$ and by~$m$ the minimum of the right ancestor (resp.~descendant) subtree of~$j$. Then for each~$k$ with~$M < k < m$, there is a directed path from~$k$ to~$j$ (resp.~from~$j$ to~$k$).
\end{lemma}

\begin{proof}
\vincent{todo}
\end{proof}

%%%%%%%%%%%%%%%%%%%%%%%%%%%%%%%%%%%%%%%

\section{Simple congruences}
\label{sec:simpleCongruences}

%%%%%%%%%%%%%%%%%%%

\subsection{Background on lattice congruences of the weak order}
\label{subsec:backgroundLatticeCongruences}

%%%%%%%%%%%%%%%%%%%

\subsection{Insertion map}
\label{subsec:insertionMap}

%%%%%%%%%%%%%%%%%%%

\subsection{Simple congruences}
\label{subsec:simpleCongruences}

\begin{definition}
\label{def:simpleCongruence}
A lattice congruence of the weak order is \defn{simple} if the minimal (for the subarc order) arcs not contained in its arc ideal are up arcs (of the form~$(a, b, {]a,b[}, \varnothing)$) or down arcs (of the form~$(a, b, \varnothing, {]a,b[})$).
\end{definition}

\begin{example}
Classical examples of simple congruences of the weak order are
\begin{itemize}
\item the trivial congruence,
\item the Sylvester congruences~\cite{Tonks, HivertNovelliThibon-algebraBinarySearchTrees},
\item the Cambrian congruences~\cite{Reading-CambrianLattices, ChatelPilaud},
\item the permutree congruences~\cite{PilaudPons-permutrees},
\item the graph associahedra congruences of~\cite{BarnardMcConville}.
\end{itemize}
A good example of a congruence which is not simple is the Baxter congruence~\cite{LawReading, Giraudo} whose quotient corresponds to diagonal rectangulations of a square.
\end{example}

The main result of this section is a simple combinatorial proof of the following statement.

\begin{theorem}[{\cite[Thm.~??]{HoangMutze} \& \cite[Thm.~??]{DemonetIyamaReadingReitenThomas}}]
The following assertions are equivalent for a lattice congruence~$\equiv$ of the weak order:
\begin{enumerate}[(i)]
\item $\equiv$ is a simple congruence,
\item the Hasse diagram of any $\equiv$-poset is a tree,
\item the Hasse diagram of the lattice quotient~$\f{S}_n/{\equiv}$ is regular (\ie constant degree),
\item the quotient fan~$\QuotientFan$ is a simplicial fan,
\item the quotientope~$\Quotientope$ is a simple polytope.
\end{enumerate}
\end{theorem}

This statement was originally proved by H.~Hoang and T.~Mütze in~\cite{HoangMutze} with a purely combinatorial (but slightly involved) proof.
In their paper on the lattice theory of torsion classes~\cite{DemonetIyamaReadingReitenThomas}, L.~Demonet, O.~Iyama, N.~Reading, I.~Reiten, and H.~Thomas provided an alternative proof in terms of quiver representation theory.
Our proof is a very elementary approach just based on the insertion map described in~\cref{subsec:insertionMap}.
In fact, we prove the following stronger statement on the $\equiv$-posets of a simple congruence~$\equiv$.

\begin{proposition}
\label{prop:simpleImpliesSeparatingTrees}
If~$\equiv$ is a simple congruence of the weak order, then all the Hasse diagram of any $\equiv$-poset is a separating tree.
\end{proposition}

\begin{proof}
\vincent{todo}
\end{proof}

In fact, we can actually characterize the $\equiv$-posets of a simple congruence~$\equiv$.
For this, we need the following two definitions.

\begin{definition}
\label{lem:edgeIsArc}
From \cref{lem:separatingTree1}, each edge in a separating tree~$T$ actually defines an arc. Namely
\begin{itemize}
\item each edge~$i \leftarrow k$ in~$T$ with~$i < k$ defines an arc~$(i, k, A, B)$ where~$A$ (resp.~$B$) is the set of nodes~$j \in {]i,k[}$ such that there is a directed path in~$T$ joining $j$ to~$k$ (resp.~$i$ to~$j$),
\item each edge~$i \to k$ in~$T$ with~$i < k$ defines an arc~$(i, k, A, B)$ where~$A$ (resp.~$B$) is the set of nodes~$j \in {]i,k[}$ such that there is a directed path in~$T$ joining $j$ to~$i$ (resp.~$k$ to~$j$).
\end{itemize}
\end{definition}

\begin{definition}
\label{def:admissibleSeparatingTrees}
Fix a (simple) lattice congruence~$\equiv$ of the weak order.
A separating tree~$T$ is \defn{$\equiv$-admissible} if
\begin{enumerate}[(i)]
\item all edges of~$T$ are allowed arcs in~$\equiv$,
\item if a node~$j$ of~$T$ has two parents, and $M$ (resp.~$m$) denotes the maximum (resp.~minimum) of the left (resp.~right) ancestor subtree of~$j$, then the up arc~$(M, m, {]M,m[}, \varnothing)$ is forbidden~in~$\equiv$,
\item if a node~$j$ of~$T$ has two children, and $M$ (resp.~$m$) denotes the maximum (resp.~minimum) of the left (resp.~right) descendant subtree of~$j$, then the down arc~$(M, m, \varnothing, {]M,m[})$ is forbidden in~$\equiv$.
\end{enumerate}
\end{definition}

\begin{lemma}
If~$T$ and~$T'$ are two $\equiv$-admissible separating trees with a common linear extension, then~${T = T'}$.
\end{lemma}

\begin{proof}
\vincent{todo}
\end{proof}

\begin{proposition}
\label{prop:admissibleSeparatingTrees}
For a simple congruence~$\equiv$ of the weak order, the $\equiv$-posets are precisely the (linear extensions of the) $\equiv$-admissible separating trees.
\end{proposition}

\begin{proof}
\vincent{todo}
\end{proof}

%%%%%%%%%%%%%%%%%%%%%%%%%%%%%%%%%%%%%%%

\section{Schröder separating trees}
\label{sec:SchroderSeparatingTrees}

%%%%%%%%%%%%%%%%%%%%%%%%%%%%%%%%%%%%%%%

\section{Hopf algebra}
\label{sec:HopfAlgebra}

%%%%%%%%%%%%%%%%%%%%%%%%%%%%%%%%%%%%%%%

\section{Quiver representation theory}
\label{subsec:representationTheory}

%%%%%%%%%%%%%%%%%%%%%%%%%%%%%%%%%%%%%%%

\addtocontents{toc}{\vspace{.1cm}}
\section*{Acknowledgments}

%%%%%%%%%%%%%%%%%%%%%%%%%%%%%%%%%%%%%%%

\bibliographystyle{alpha}
\bibliography{separatingTrees}
\label{sec:biblio}

\end{document}
